% ==========================================
% BAB II STUDI LITERATUR
% ==========================================
\chapter{STUDI LITERATUR}
\label{chap:studi-literatur}

% Desain Interaksi
\section{Desain Interaksi}
Desain interaksi merupakan disiplin yang berfokus pada perancangan sistem dan produk interaktif dengan memperhatikan bagaimana pengguna berinteraksi, berkomunikasi, dan mencapai tujuan mereka secara efektif \parencite{ixdf2016}. \textcite{sharp2019} menjelaskan bahwa desain interaksi mencakup lima dimensi interaksi: kata-kata (1D), representasi visual (2D), objek/ruang fisik (3D), waktu (4D), dan perilaku (5D).
\subsection{Prinsip-Prinsip Desain Interaksi}
\begin{enumerate}
\item   \textbf{\textit{Visibility}}\\
        \textit{Visibility} memastikan bahwa fitur, menu, dan informasi penting terlihat jelas sehingga pengguna mengetahui pilihan apa yang dapat dilakukan tanpa kebingungan. \textcite{norman2013} menjelaskan bahwa rendahnya visibilitas membuat pengguna tidak dapat memahami fungsi sistem, sehingga meningkatkan beban kognitif.
\item   \textbf{\textit{Feedback}}\\
        \textit{Feedback} memberikan respons langsung atas tindakan pengguna, baik berupa notifikasi, perubahan visual, maupun pesan status. \textcite{nielsen2024} menekankan bahwa \textit{feedback} mencegah ketidakpastian dan meningkatkan kepercayaan pengguna terhadap sistem.
\item   \textbf{\textit{Consistency}}\\
        \textit{Consistency} menjaga keseragaman desain, bahasa, dan perilaku sistem agar pengguna dapat memprediksi hasil dari suatu tindakan. \textcite{shneiderman1992} menjelaskan bahwa konsistensi mempermudah pembelajaran dan mengurangi kesalahan.
\item   \textbf{\textit{Affordance}}\\
        \textit{Affordance} merujuk pada petunjuk visual yang menunjukkan bagaimana suatu elemen dapat digunakan. \textcite{norman2013} menyatakan bahwa \textit{affordance} yang baik membantu pengguna memahami cara kerja fitur tanpa instruksi tambahan.
\item   \textbf{\textit{Constraints}}\\
        \textit{Constraints} membatasi pilihan pengguna untuk mencegah kesalahan dan mengarahkan interaksi ke jalur yang tepat dalam penggunaan antarmuka \parencite{norman2013}. \textcite{lidwell2010} membagi \textit{constraints} ke dalam dua kategori, yaitu fisik dan psikologis, yang memengaruhi cara pengguna menafsirkan dan memilih tindakan di antarmuka.
\end{enumerate}

\subsection{Tipe Interaksi}
Menurut \textcite{sharp2019}, terdapat lima tipe utama interaksi yang menjelaskan bagaimana pengguna berhubungan dengan sistem digital.
\begin{enumerate}
\item   \textbf{\textit{Instructing}}\\
        \textit{Instructing} adalah bentuk interaksi ketika pengguna memberikan perintah langsung kepada sistem, seperti mengetikkan \textit{command}, memilih menu, menekan tombol, menggunakan \textit{gesture multitouch}, atau mengucapkan perintah suara. Bentuk interaksi ini umum digunakan karena cepat dan efisien untuk menjalankan tugas-tugas sederhana \parencite{sharp2019}.
\item   \textbf{\textit{Conversing}}\\
        \textit{Conversing} terjadi ketika pengguna melakukan dialog dengan sistem, baik melalui teks maupun suara, dan sistem merespons layaknya percakapan dua arah. Contohnya \textit{chatbot} atau \textit{voice assistant} yang memberikan jawaban berdasarkan input pengguna \parencite{sharp2019}.
\item   \textbf{\textit{Manipulating}}\\
        \textit{Manipulating} melibatkan interaksi langsung dengan objek fisik atau virtual, misalnya menggeser, membuka, memutar, atau menempatkan objek di layar. Interaksi ini memanfaatkan pengetahuan alami pengguna terhadap objek, sehingga terasa intuitif \parencite{sharp2019}.
\item   \textbf{\textit{Exploring}}\\
        \textit{Exploring} memungkinkan pengguna bergerak dalam lingkungan fisik atau virtual, seperti menjelajahi peta digital, ruang 3D, VR, atau AR. Sistem memberikan representasi lingkungan yang dapat dinavigasi dan memungkinkan pengguna memahami konteks ruang melalui eksplorasi aktif \parencite{sharp2019}.
\item   \textbf{\textit{Responding}}\\
        \textit{Responding} terjadi ketika sistem memulai interaksi terlebih dahulu, misalnya menampilkan notifikasi berbasis lokasi atau rekomendasi otomatis, dan pengguna memutuskan apakah akan merespons atau mengabaikannya \parencite{sharp2019}.
\end{enumerate}

% User Experience
\section{\textit{User Experience}}
\textit{User experience} (UX) adalah keseluruhan pengalaman yang dialami pengguna dalam berinteraksi dengan produk digital, mulai dari tujuan bisnis hingga pengalaman visual yang mereka rasakan \parencite{garrett2011}. \textcite{iso9241-210} menegaskan bahwa UX tidak hanya tentang kemudahan penggunaan, tetapi juga makna, nilai, serta kepuasan yang diperoleh pengguna secara keseluruhan. Dengan demikian, UX berfokus pada bagaimana sebuah sistem dapat memberikan pengalaman yang efektif, menyenangkan, dan bermakna bagi penggunanya.
\subsection{Model UX Garrett}
Model \textit{The Elements of User Experience} yang dikembangkan oleh \textcite{garrett2011} menjelaskan bahwa pengalaman pengguna tersusun dari lima lapisan yang saling berkaitan dan harus dirancang secara bertahap untuk menghasilkan produk digital yang efektif, mudah digunakan, dan selaras dengan kebutuhan pengguna. Lapisan ini terdiri dari \textit{strategy}, \textit{scope}, \textit{structure}, \textit{skeleton}, dan \textit{surface} yang membangun alur logis mulai dari tujuan konseptual hingga tampilan visual yang digunakan pengguna secara langsung. 
\begin{enumerate}
\item   \textbf{\textit{Strategy}}\\
        Lapisan paling dasar yang berfokus pada tujuan bisnis serta kebutuhan pengguna yang ingin dipenuhi oleh produk digital \parencite{garrett2011}.
\item   \textbf{\textit{Scope}}\\
        Menentukan ruang lingkup produk yang mencakup kebutuhan fungsional serta persyaratan konten yang harus disediakan \parencite{garrett2011}.
\item   \textbf{\textit{Structure}}\\
        Mengatur arsitektur informasi dan alur interaksi agar pengguna dapat menyelesaikan tugas dengan cara yang logis dan konsisten \parencite{garrett2011}.
\item   \textbf{\textit{Skeleton}}\\
        Berfokus pada perancangan tata letak antarmuka, penempatan elemen, navigasi, serta penyusunan \textit{wireframe} agar interaksi menjadi efisien \parencite{garrett2011}.
\item   \textbf{\textit{Surface}}\\
        Lapisan visual yang meliputi warna, tipografi, ikonografi, dan gaya estetika untuk menyampaikan informasi dan menciptakan kesan pengalaman \parencite{garrett2011}.
\end{enumerate}

\subsection{\textit{User Experience Goals}}
\textit{Goals} dari UX berfokus pada pengalaman subjektif yang ingin dicapai saat pengguna berinteraksi dengan sistem. Menurut \textcite{sharp2019}, tujuan ini mencakup kualitas emosional seperti kesenangan, kenyamanan, rasa percaya diri, hingga kepuasan secara holistik. Penetapan \textit{UX goals} membantu memastikan bahwa desain tidak hanya berfungsi, tetapi juga memberikan pengalaman yang positif, bermakna, dan \textit{engaging} bagi pengguna.

Berikut adalah beberapa \textit{UX goals} pada pembangunan desain antarmuka yang dikemukakan oleh \textcite{sharp2019}.
\begin{table}[h]
  \caption{Daftar \textit{UX Goals}}
  \label{tbl:ux_goals}
  \begin{tabular}{|l|l|l|}
  \hline
  \textit{Satisfying} & \textit{Helpful} & \textit{Fun} \\ 
  \hline
  \textit{Enjoyable} & \textit{Motivating} & \textit{Provocative} \\ 
  \hline
  \textit{Engaging} & \textit{Challenging} & \textit{Surprising} \\ 
  \hline
  \textit{Pleasurable} & \textit{Enhancing sociability} & \textit{Rewarding} \\ 
  \hline
  \textit{Exciting} & \textit{Supporting creativity} & \textit{Emotionally fulfilling} \\ 
  \hline
  \textit{Entertaining} & \textit{Cognitively stimulating} & \textit{Experiencing flow} \\ 
  \hline
  \end{tabular}
\end{table}

% Usability
\section{\textit{Usability}}
\textcite{iso9241-210} menjelaskan \textit{usability} sebagai seberapa efektif, efisien, dan menyenangkan suatu sistem digunakan untuk mencapai tujuan pengguna. Dalam dunia UI/UX, konsep ini menjadi dasar agar sebuah antarmuka tidak hanya berfungsi, tetapi juga terasa mudah dan nyaman digunakan.
\subsection{\textit{Usability Goals}}
\textit{Usability goals} berfokus pada pencapaian efektivitas, efisiensi, dan kepuasan dalam penggunaan sistem. \textcite{sharp2019} menegaskan bahwa tujuan ini penting untuk memastikan bahwa pengguna dapat menyelesaikan tugas tanpa hambatan, belajar menggunakan sistem dengan cepat, serta meminimalisir kesalahan. Dengan menetapkan \textit{usability goals}, proses desain menjadi lebih terarah untuk menghasilkan antarmuka yang mudah dipahami dan dioperasikan. Berikut adalah \textit{usability goals} menurut \textcite{sharp2019}.
\begin{enumerate}
\item   \textit{Effectiveness}\\
        Sistem harus memungkinkan pengguna mencapai tujuan yang diinginkan secara akurat dan tanpa hambatan.
\item   \textit{Efficiency}\\
        Pengguna dapat membuat atau mengedit rencana perjalanan dengan cepat melalui alur yang ringkas, dukungan visual, dan interaksi minimal.
\item   \textbf{\textit{Safety}}\\
        Sistem melindungi pengguna dari kesalahan, baik dengan mencegah terjadinya error maupun memudahkan pemulihan jika kesalahan terjadi.
\item   \textbf{\textit{Utility}}\\
        Sistem menyediakan fungsi yang relevan dan dibutuhkan pengguna untuk menyelesaikan berbagai tugas dengan lengkap.
\item   \textbf{\textit{Learnability}}\\
        Pengguna baru dapat mempelajari cara menggunakan sistem dengan mudah dan memahami fungsinya dalam waktu singkat.
\item   \textbf{\textit{Memorability}}\\
        Pengguna yang kembali setelah tidak mengakses fitur untuk beberapa waktu tetap dapat mengingat alur penggunaan tanpa harus mempelajari ulang.
\end{enumerate}

\subsection{\textbf{\textit{Usability Testing}}}
Evaluasi \textit{usability} diperlukan untuk memahami sejauh mana antarmuka aplikasi dapat digunakan secara efektif, efisien, dan memuaskan oleh pengguna \parencite{iso9241-210}. Evaluasi ini penting karena pengguna sering berinteraksi dengan informasi dan tugas yang kompleks sehingga sistem harus benar-benar mendukung penggunaan yang optimal. Oleh karena itu, penelitian ini menggunakan metode \textit{user-based quantitative evaluation}, yaitu \textit{Single Ease Question} (SEQ), \textit{System Usability Scale} (SUS), dan \textit{Task Completion Time} (TCT) untuk menilai tingkat \textit{usability} secara objektif.
\subsubsection{\textbf{\textit{Single Ease Question}} (SEQ)}
\textit{Single Ease Question} adalah cara sederhana untuk mengetahui seberapa mudah sebuah tugas dirasakan oleh pengguna. Setelah menyelesaikan satu tugas, pengguna perlu menjawab satu pertanyaan: “Seberapa mudah tugas ini untuk dilakukan?” dengan skala 1 sampai 7. Meskipun hanya terdiri dari satu pertanyaan, SEQ cukup sensitif untuk melihat bagian mana dari alur yang masih membingungkan. Karena itu, SEQ sangat membantu dalam mengidentifikasi titik-titik masalah pada fitur dengan langkah-langkah yang kompleks.
\subsubsection{\textbf{\textit{System Usability Scale}} (SUS)}
\textit{System Usability Scale} merupakan instrumen evaluasi \textit{usability} yang dikembangkan oleh \textcite{brooke1996}, terdiri dari 10 pernyataan yang dinilai menggunakan skala \textit{Likert} 5 poin. SUS memberikan skor kuantitatif yang reliabel dan dapat dibandingkan secara universal dengan berbagai sistem lain. Instrumen ini banyak digunakan dalam penelitian karena sifatnya yang sederhana, cepat, namun tetap menghasilkan validitas tinggi \parencite{bangor2008}. Berikut adalah 10 pertanyaan SUS.
\begin{enumerate}
\item   Saya rasa saya akan sering menggunakan sistem ini.
\item   Saya merasa sistem ini terlalu rumit.
\item   Saya pikir sistem ini mudah digunakan.
\item   Saya merasa akan membutuhkan bantuan dari orang teknis untuk dapat menggunakan sistem ini.
\item   Saya merasa berbagai fungsi dalam sistem ini terintegrasi dengan baik.
\item   Saya pikir ada terlalu banyak ketidakkonsistenan dalam sistem ini.
\item   Saya membayangkan sebagian besar orang akan belajar menggunakan sistem ini dengan cepat.
\item   Saya merasa sistem ini sangat merepotkan untuk digunakan.
\item   Saya merasa percaya diri dalam menggunakan sistem ini.
\item   Saya perlu mempelajari banyak hal sebelum dapat mulai menggunakan sistem ini.
\end{enumerate}
Untuk pernyataan bernomor ganjil (pernyataan positif), skor dihitung dengan mengurangkan nilai jawaban dengan 1. Sebaliknya, untuk pernyataan bernomor genap (pernyataan negatif), skor dihitung dengan cara 5 dikurangi nilai jawaban. Nilai dari seluruh pernyataan kemudian dijumlahkan dan dikalikan dengan 2,5 sehingga menghasilkan skor total pada rentang 0 hingga 100. Skor ini digunakan untuk menilai tingkat \textit{usability} secara keseluruhan, dimana skor sekitar 68 dianggap sebagai nilai rata-rata dan skor di atasnya menunjukkan \textit{usability} yang lebih baik \parencite{brooke1996}.
\subsubsection{\textbf{\textit{Task Completion Time}} (TCT)}
\textit{Task Completion Time} mengukur berapa lama pengguna membutuhkan waktu untuk menyelesaikan suatu tugas. Semakin cepat waktu yang dibutuhkan, semakin efisien desain tersebut mendukung pengguna. Jika waktu penyelesaian lama atau pengguna tampak ragu-ragu, hal itu dapat menjadi indikator bahwa alur atau tampilan antarmuka perlu diperbaiki atau disederhanakan. Karena bersifat objektif, TCT sering digunakan untuk melengkapi hasil SEQ dan SUS sehingga analisis \textit{usability} menjadi lebih menyeluruh.

% Cognitive Aspects
\section{\textbf{Aspek Kognitif dalam \textit{Human Computer Interaction} (HCI)}}
Aspek kognitif dalam \textit{Human–Computer Interaction} (HCI) berfokus pada bagaimana pengguna memproses informasi, mengambil keputusan, dan menyelesaikan tugas melalui berinteraksi dengan sistem digital. Memahami proses kognitif pengguna menjadi penting karena keterbatasan memori kerja manusia dapat memengaruhi efektivitas penggunaan antarmuka \parencite{sweller1988}. Selain itu, desain antarmuka yang tidak selaras dengan cara kerja kognitif pengguna berpotensi meningkatkan beban mental, memperlambat penyelesaian tugas, dan menurunkan kepuasan pengguna \parencite{norman2013}.
\subsection{\textbf{\textit{Cognitive Load Theory}} (CLT)}
\textit{Cognitive Load Theory}, yang dikembangkan oleh \textcite{sweller1988,sweller2011}, menjelaskan bahwa kapasitas memori kerja manusia terbatas. Beban kognitif yang berlebihan dapat menghambat pemrosesan informasi, pengambilan keputusan, dan performa pengguna. CLT membagi beban kognitif menjadi tiga jenis:
\begin{enumerate}
\item   \textbf{\textit{Intrinsic Cognitive Load}}\\
        \textit{Intrinsic cognitive load} adalah beban kognitif yang berasal dari kompleksitas suatu tugas atau materi yang harus diproses pengguna \parencite{sweller2011}. Beban ini bergantung pada tingkat kesulitan informasi dan jumlah elemen yang harus dipahami secara bersamaan.
\item   \textbf{\textit{Extraneous Cognitive Load}}\\
        \textit{Extraneous cognitive load} muncul akibat cara penyajian informasi atau desain sistem yang kurang optimal \parencite{kirschner2018}. Beban ini dapat meningkat apabila antarmuka membingungkan, informasi disajikan secara tidak jelas, atau istilah yang digunakan tidak familiar bagi pengguna. Desain yang baik bertujuan untuk meminimalkan beban ini.
\item   \textbf{\textit{Germane Cognitive Load}}\\
        \textit{Germane cognitive load} merupakan beban yang dialokasikan untuk proses pembelajaran, pemahaman, dan pembentukan skema mental pengguna \parencite{sweller2011}. Beban ini bersifat positif karena mendukung peningkatan pengetahuan dan keterampilan selama pengguna berinteraksi dengan sistem.
\end{enumerate}

\subsection{\textbf{Pengukuran \textit{Cognitive Load}}}
Mengukur \textit{cognitive load} membantu mengetahui seberapa besar beban mental yang dialami pengguna saat menyelesaikan tugas. Evaluasi ini penting untuk mengidentifikasi bagian antarmuka yang membingungkan atau tidak efisien. Menurut \parencite{kosch2023}, metode pengukuran yang umum digunakan dibagi menjadi dua kategori yaitu subjektif dan \textit{behavioral}. 
\subsubsection{\textbf{\textit{NASA Task Load Index}} (NASA-TLX)}
NASA-TLX adalah instrumen paling populer untuk mengukur beban kerja subjektif dalam HCI \parencite{hart1988}. Alat ini menilai enam dimensi berikut:
\begin{enumerate}
\item   \textit{Mental Demand}: Seberapa banyak aktivitas mental yang dibutuhkan.
\item   \textit{Physical Demand}: Sejauh mana aktivitas fisik diperlukan.
\item   \textit{Temporal Demand}: Tekanan waktu saat menyelesaikan tugas.
\item   \textit{Performance}: Persepsi pengguna terhadap keberhasilan mereka.
\item   \textit{Effort}: Jumlah usaha yang dibutuhkan untuk mencapai performa.
\item   \textit{Frustration Level}: Tingkat stres dan frustrasi pengguna.
\end{enumerate}
Sebagai metode subjektif, NASA-TLX mengukur persepsi pengguna terhadap beban kerja, bukan beban kerja objektif. Instrumen ini membantu untuk menangkap pengalaman mental pengguna secara langsung dari sudut pandang mereka.
\subsubsection{\textbf{\textit{Task Performance}}}
Selain pengukuran subjektif, \textit{cognitive load} juga dapat dianalisis melalui indikator perilaku saat pengguna menyelesaikan tugas \parencite{nielsen1993}. Indikator umum meliputi:
\begin{enumerate}
\item   Waktu penyelesaian tugas: Waktu yang lebih lama dapat menandakan beban kognitif tinggi.
\item   Jumlah kesalahan (\textit{errors}): \textit{Error rate} tinggi menunjukkan \textit{extraneous load} yang tinggi.
\item   Jumlah langkah/klik: Langkah yang tidak perlu meningkatkan beban mental.
\item   \textit{Navigational deviation}: Penyimpangan dari jalur optimal menunjukkan kebingungan pengguna.
\end{enumerate}
\textit{Behavioral metrics} seperti ini memberikan data objektif yang melengkapi NASA-TLX, sehingga evaluasi \textit{usability} dapat menangkap performa kognitif pengguna secara lebih menyeluruh.

% OTA
\section{\textbf{\textit{Online Travel Agency} (OTA)}}
\textit{Online Travel Agency} (OTA) merupakan platform digital yang menyediakan layanan pemesanan perjalanan, seperti tiket transportasi, hotel, dan paket wisata \parencite{mulyana2023}. OTA berfungsi sebagai perantara antara penyedia layanan perjalanan dan pengguna dengan menyediakan fitur pencarian, perbandingan, serta pemesanan dalam satu sistem yang terintegrasi. Menurut \textcite{buhalis2008}, OTA merupakan bagian dari \textit{e-tourism} yang memungkinkan pengguna memperoleh informasi perjalanan serta melakukan transaksi secara mandiri melalui teknologi digital. Dalam industri pariwisata modern, OTA menjadi kanal distribusi yang dominan karena menawarkan akses informasi yang luas, transparansi harga, dan kemudahan proses pemesanan \parencite{xiang2014}.

Di Indonesia, penggunaan OTA berkembang pesat seiring dengan adopsi teknologi \textit{mobile} dan meningkatnya minat masyarakat terhadap pemesanan perjalanan berbasis digital. Beberapa platform OTA yang banyak digunakan di Indonesia antara lain Traveloka, Tiket.com, dan Agoda. Kehadiran berbagai OTA ini menunjukkan tingginya permintaan terhadap layanan pemesanan yang praktis, fleksibel, serta terintegrasi dengan metode pembayaran lokal seperti transfer bank dan dompet digital. Selain itu, OTA di Indonesia umumnya menawarkan pilihan layanan yang luas, mencakup transportasi domestik, hotel, aktivitas wisata, dan promo berbasis kebutuhan pasar lokal.

Secara umum, layanan yang disediakan oleh OTA adalah sebagai berikut:
\begin{enumerate}
\item   Pemesanan Transportasi\\
        OTA menyediakan layanan pemesanan berbagai moda transportasi, seperti pesawat, kereta, bus, hingga transportasi antarkota lainnya. Pengguna dapat membandingkan harga, jadwal, durasi perjalanan, serta kebijakan bagasi secara langsung dalam satu platform.
\item   Pemesanan Akomodasi\\
        Layanan ini mencakup pemesanan hotel, vila, apartemen, atau \textit{homestay}. OTA biasanya menyediakan fitur seperti filter fasilitas, ulasan pengguna, foto properti, dan perbandingan harga guna membantu pengguna memilih akomodasi yang sesuai kebutuhan.
\item   Tiket Atraksi dan Aktivitas Wisata\\
        Banyak OTA menawarkan pemesanan tiket tempat wisata, tur lokal, aktivitas rekreasi, dan pengalaman perjalanan lain. Fitur ini memudahkan pengguna merencanakan kegiatan selama perjalanan.
\item   Paket Wisata Terintegrasi\\
        Beberapa OTA menyediakan \textit{bundling} berupa paket perjalanan yang menggabungkan transportasi, akomodasi, hingga aktivitas tertentu. Paket ini membantu pengguna menghemat biaya dan mengurangi kompleksitas dalam perencanaan perjalanan.
\item   Penyewaan Kendaraan\\
        OTA juga menawarkan layanan pemesanan mobil sewaan, motor, atau kendaraan lainnya, baik dengan sopir maupun tanpa sopir. Penyediaan informasi detail seperti jenis kendaraan, harga, dan kebijakan penggunaan membantu pengguna membandingkan pilihan secara mudah.
\item   Mode Pembayaran dan Pengelola Pesan\\
        OTA menyediakan sistem pembayaran digital yang aman dan beragam, seperti dengan kartu kredit, transfer bank, atau dompet digital. Selain itu, pengguna dapat mengelola pesanan melalui fitur \textit{e-ticket}, \textit{refund}, \textit{reschedule}, atau perubahan data perjalanan secara langsung di dalam aplikasi.
\item   Informasi Perjalanan dan Dukungan Pelanggan\\
        Layanan tambahan seperti ulasan pengguna, panduan destinasi, notifikasi penerbangan, serta \textit{customer service} membantu meningkatkan pengalaman pengguna dan mengurangi ketidakpastian selama proses perjalanan.
\end{enumerate}

\section{Perencanaan Perjalanan}
Perencanaan perjalanan merupakan proses di mana wisatawan mengatur dan mempersiapkan seluruh aspek perjalanan sebelum keberangkatan, termasuk pemilihan destinasi, transportasi, akomodasi, aktivitas, dan pengelolaan waktu serta anggaran \parencite{jamal2009}. Proses ini penting karena perencanaan yang matang dapat meningkatkan efisiensi, meminimalkan risiko, serta meningkatkan kepuasan pengalaman wisatawan.
\subsection{Model Pengalaman Wisata}
\textcite{clawson2011} mengemukakan lima tahap pengalaman wisata, yaitu \textit{anticipation}, \textit{travel to site}, \textit{on-site experience}, \textit{return travel}, dan \textit{recollection}. Pada fase \textit{anticipation}, wisatawan mulai membayangkan perjalanan dan melakukan perencanaan awal, termasuk menentukan tujuan dan mempersiapkan kebutuhan dasar. Fase \textit{travel to site} mencakup perjalanan menuju destinasi, di mana perencanaan transportasi, estimasi waktu tempuh, dan persiapan dokumen menjadi penting. Selama \textit{on-site experience}, wisatawan menikmati berbagai aktivitas di lokasi tujuan, termasuk eksplorasi tempat, interaksi sosial, dan partisipasi dalam kegiatan lokal. Fase \textit{return travel} adalah perjalanan pulang ke tempat asal, yang memerlukan pengaturan waktu dan logistik agar proses kepulangan berjalan lancar. Terakhir, fase {recollection} terjadi setelah perjalanan selesai, ketika wisatawan merefleksikan pengalaman, menyimpan kenangan, dan seringkali membagikan cerita atau ulasan perjalanan kepada orang lain.
\subsection{Tahapan Perjalanan Modern}
Model perjalanan modern mencakup lima tahapan utama \parencite{shirgwin2024}:
\begin{enumerate}
\item   \textit{Dreaming}: Tahap di mana calon wisatawan mulai membayangkan perjalanan yang ingin dilakukan, mencari inspirasi, dan mempertimbangkan pengalaman yang diinginkan.
\item   \textit{Researching}: Pengumpulan informasi rinci mengenai destinasi, akomodasi, transportasi, dan aktivitas yang tersedia.
\item   \textit{Booking}: Tahap pemesanan tiket, akomodasi, atau paket wisata berdasarkan informasi yang telah dikumpulkan.
\item   \textit{Experiencing}: Tahap pelaksanaan perjalanan di lokasi tujuan, termasuk menikmati aktivitas, menjelajahi tempat, dan berpartisipasi dalam kegiatan lokal.
\item   \textit{Sharing}: Tahap refleksi setelah perjalanan selesai, di mana wisatawan membagikan pengalaman, ulasan, atau rekomendasi kepada teman dan komunitas.
\end{enumerate}

\subsection{\textit{Travel Constraints}}
Dalam merencanakan perjalanan, wisatawan sering menghadapi berbagai hambatan yang dapat mempengaruhi keputusan dan kepuasan mereka \parencite{clawson2011}.
\begin{enumerate}
\item   \textbf{textit{Intrapersonal}}\\
        Hambatan yang berasal dari diri individu, seperti motivasi, minat, atau preferensi pribadi. Misalnya, seseorang mungkin ingin berwisata ke tempat tertentu, tetapi minat dan kenyamanan pribadi dapat memengaruhi pilihan aktivitas atau durasi perjalanan.
\item   \textbf{\textit{Interpersonal}}\\
        Hambatan yang muncul dari interaksi dengan anggota kelompok atau keluarga. Perbedaan preferensi, jadwal, atau kemampuan anggota kelompok seringkali memerlukan kompromi dan koordinasi ekstra agar perjalanan tetap berjalan lancar.
\item   \textbf{\textit{Structural}}\\
        Hambatan yang terkait dengan faktor eksternal seperti waktu, biaya, atau aksesibilitas destinasi. Keterbatasan anggaran, jarak tempuh yang jauh, atau transportasi yang terbatas dapat membatasi opsi perjalanan dan mempengaruhi perencanaan aktivitas.
\end{enumerate}

\section{Penelitian Terdahulu}
Berbagai penelitian telah dilakukan untuk mengatasi kompleksitas dan tantangan dalam perencanaan perjalanan wisata melalui pendekatan teknologi dan desain antarmuka. Fokus utama dari studi-studi ini berkisar pada peningkatan \textit{usability}, penerapan metode desain yang berpusat pada pengguna, hingga integrasi kecerdasan buatan (\textit{Artificial Intelligence}) untuk personalisasi.

Dalam aspek desain interaksi dan pengalaman pengguna, \textcite{rahadiani2025} menyoroti masalah inefisiensi dan beban emosional negatif (\textit{overwhelmed}) yang dialami pengguna saat membuat itinerari. Penelitian ini menerapkan pendekatan \textit{Activity-Centered Design} (ACD) yang dikaitkan dengan tiga level \textit{Emotional Design} dari Don Norman (\textit{visceral}, \textit{behavioral}, \textit{reflective}). Hasilnya, desain ulang fitur pada dimensi \textit{mobile design} berhasil meningkatkan skor \textit{System Usability Scale} (SUS) secara drastis hingga 90, menurunkan waktu penyelesaian tugas menjadi rata-rata 182 detik, serta mencapai keterlibatan emosional pengguna pada level reflektif. 

Sejalan dengan pendekatan \textit{user-centric}, \textcite{pratiwi2023} mengembangkan aplikasi "Trinity" menggunakan metode \textit{Design Thinking}. Penelitian ini berangkat dari kesulitan wisatawan menyusun rencana praktis, dan menghasilkan prototipe dengan predikat "\textit{Excellent}" pada pengujian SUS, yang membuktikan bahwa tahapan empathize hingga test sangat krusial untuk memenuhi kebutuhan pengguna.

Di sisi pengembangan sistem dan algoritma, pendekatan teknis juga terus berkembang. \textcite{wardhana2021} mengeksplorasi pengembangan aplikasi web agenda perjalanan menggunakan metode \textit{User Experience Lifecycle}, sementara \textcite{worndl2016} mengajukan model abstrak untuk masalah desain perjalanan wisata yang berfokus pada rekomendasi urutan kunjungan kota dan kombinasi wilayah perjalanan. Lalu yang terbaru, \textcite{sharma2025} memperkenalkan "DestinAI", sebuah sistem perencanaan perjalanan berbasis AI dan \textit{Natural Language Processing} (NLP). Penelitian ini merespons fragmentasi alat pemesanan tradisional yang menyebabkan kelelahan pengambilan keputusan (\textit{decision fatigue}). Dengan memanfaatkan algoritma \textit{machine learning} dan data \textit{real-time}, DestinAI menawarkan kemampuan adaptasi dinamis dan personalisasi yang mendalam, menetapkan tolak ukur baru dalam sistem perencanaan perjalanan cerdas.

Berikut adalah tabel perbandingan penelitian terdahulu yang relevan dengan pengembangan fitur \textit{trip planner}:
\begin{longtable}{|c|p{2.2cm}|p{2.5cm}|p{2.3cm}|p{2.2cm}|p{3.2cm}|}
    \caption{Perbandingan Penelitian Terdahulu}
    \label{tab:penelitian_terdahulu}
    \endfirsthead
    
    \multicolumn{6}{|c|}{\textbf{\textit{Tabel 1.1 Perbandingan Penelitian Terdahulu (Lanjutan)}}} \\
    \hline
    \textbf{No} & \textbf{Peneliti (Tahun)} & \textbf{Judul / Topik} & \textbf{Metode} & \textbf{Temuan Utama} & \textbf{Kesenjangan (\textit{Gap}) / Keterbatasan} \\
    \hline
    \endhead 

    \hline
    \textbf{No} & \textbf{Peneliti (Tahun)} & \textbf{Judul / Topik} & \textbf{Metode} & \textbf{Temuan Utama} & \textbf{Kesenjangan (\textit{Gap}) / Keterbatasan} \\
    \hline
    
    1 & 
    \textbf{\textcite{rahadiani2025}} & 
    Mobile Design pada Desain Interaksi Fitur Itinerari Otomatis & 
    Activity-Centered Design (ACD) \& Emotional Design (Norman's 3 Levels) & 
    (a) Skor SUS naik jadi 90 \newline (b) Waktu tugas turun drastis (2.726s ke 182s) \newline (c) Mencapai level \textit{reflective} & 
    Fokus sangat spesifik pada aspek emosional. Perlu eksplorasi lebih lanjut mengenai pengukuran beban kognitif (\textit{cognitive load}) yang mendalam pada tugas kompleks. \\
    \hline
    
    2 & 
    \textbf{\textcite{sharma2025}} & 
    User-adapted travel planning system for personalized schedule recommendation & 
    Travel schedule planning algorithm \& Feedback mechanism & 
    (a) Peningkatan kepuasan pengguna signifikan \newline (b) Performansi tinggi pada penyesuaian jadwal & 
    Fokus pada validitas algoritma rekomendasi. Tidak membahas aspek psikologis mendalam seperti \textit{Emotional Design} atau pengukuran spesifik \textit{Cognitive Load} pada antarmuka. \\
    \hline
    
    3 & 
    \textbf{\textcite{pratiwi2023}} & 
    Implementasi Design Thinking Perancangan Aplikasi Itinerary & 
    Design Thinking (Empathize - Test) & 
    (a) Prototipe "Trinity" \newline (b) Skor SUS predikat 'Excellent' & 
    Menggunakan pendekatan umum (\textit{Design Thinking}). Belum menerapkan metode spesifik untuk otomatisasi dan tidak mengukur level emosional secara mendalam. \\
    \hline
    
    4 & 
    \textbf{\textcite{wardhana2021}} & 
    Pengembangan Aplikasi Web Agenda Perjalanan Wisata & 
    User Experience Lifecycle & 
    (a) Sistem berbasis web untuk agenda terstruktur \newline (b) Evaluasi siklus hidup UX & 
    Studi dilakukan pada platform \textbf{Web}. Karakteristik interaksi dan kendala layar berbeda dengan aplikasi \textbf{Mobile} yang menjadi fokus saat ini. \\
    \hline
    
    5 & 
    \textbf{\textcite{worndl2016}} & 
    Solving tourist trip design problems from a user's perspective & 
    Abstract Model for Trip Design & 
    (a) Model rekomendasi urutan kunjungan \newline (b) Rekomendasi wilayah komposit & 
    Fokus pada model teoritis dan logika algoritma rekomendasi saja. Tidak membahas implementasi desain UI/UX modern pada perangkat seluler. \\
    \hline
    
\end{longtable}


% \begin{figure}[h] % pilihan opsi yang disarankan: t = top, b = bottom, h = here
% 	\centering
%   \captionsetup{justification=centering}
%     	\includegraphics[width=0.7\textwidth]{image/gambar1.png}
% 	\caption{Contoh gambar jaringan}
% 	\label{gambar:jaringan}
% \end{figure}

% Gambar umumnya tidak jelas atau kabur jika gambar tersebut:
% \begin{enumerate}[a.]
%   \item diperoleh dari hasil cropping pada suatu halaman buku atau situs web;
%   \item hasil pembesaran gambar yang gambar aslinya sebenarnya berukuran kecil; atau
%   \item disimpan dalam resolusi kecil
% \end{enumerate}
% Ketidakjelasan gambar ini dapat dilihat pada garis-garis diagram yang tidak tegas dan tulisan-tulisan dalam gambar yang tampak kabur dan kurang jelas terbaca.

% Untuk mendapatkan gambar yang tidak kabur (\textit{blur}), langkah-langkah berikut dapat digunakan:
% \begin{enumerate}[(a)]
% \item Gambar yang didapat di suatu pustaka atau referensi sebaiknya digambar ulang, misalnya menggunakan PowerPoint, Canva, Figma, draw.io, atau yang lainnya.
% \item Jika diagram atau ilustrasi digambar menggunakan draw.io, saat gambar disimpan ke format PNG atau JPG (\textit{export as}), lakukan \textit{zoom} ke minimal 300\% (\textit{the default value is} 100\%). 
% \item Jika diagram digambar dengan menggunakan PowerPoint, gambar dapat langsung di-\textit{copy-paste} ke Word.
% \end{enumerate}

% \subsection{Tabel}
% Tabel ada dua jenis, yaitu tabel yang bisa termuat dalam satu halaman dan tabel yang sangat panjang sehingga tidak muat dalam satu halaman.
% \subsubsection{Tabel yang Muat dalam Satu Halaman}
% Contoh tabel dapat dilihat pada Tabel \ref{tbl:harga1} dan \ref{tbl:harga2}. Tabel dan judulnya dibuat rata kiri dan judul tabel diletakkan di atas tabel. Usahakan tabel dapat ditulis dalam satu halaman, tidak terpotong ke halaman berikutnya.

% \begin{table}[t] % pilihan opsi yang disarankan: t = top, b = bottom, h = here
%   \begin{tabular}{ | p{2cm} | p{2cm} | p{3cm} |}
% 	\hline
% 	Nama 	& Satuan 		& Harga \\
% 	\hline
% 	Buku 	& Exemplar	& 25000 \\
% 	Komputer	& Unit		& 2500000 \\
% 	Pensil	& Buah		& 118900 \\
% 	\hline
% 	\end{tabular}
% \caption{Tabel harga bahan pokok}
% \label{tbl:harga1}
% \end{table}



% \begin{table}[t] % pilihan opsi yang disarankan: t = top, b = bottom, h = here
% 	\begin{tabular}{ | l | c | r | }
% 	\hline
% 	Nama 	& Satuan 		& Harga \\
% 	\hline
% 	Buku 	& Exemplar	& 25000 \\
% 	Komputer	& Unit		& 2500000 \\
% 	Pensil	& Buah		& 118900 \\
% 	\hline
% 	\end{tabular}
% \caption{Tabel harga bahan sekunder}
% \label{tbl:harga2}
% \end{table}

% % -- Example of importing table from external file --
% \subsubsection{Mengimpor Tabel dari Berkas Eksternal}

% Tabel \ref{tbl:harga3} diimpor dari berkas eksternal \textit{table/tabel1.tex} menggunakan perintah \textit{input}. 
% Dengan demikian, jika tabel tersebut perlu diubah, cukup mengubah pada berkas eksternal tersebut tanpa perlu mengubah pada berkas utama ini.

% \input table/tabel1.tex


% % -- Example of long table --
% \subsubsection{Tabel yang Sangat Panjang}
% Jika tabel terlalu panjang sehingga tidak muat dalam satu halaman, gunakan paket 
% \textit{longtable} untuk membuat tabel yang dapat terpotong ke halaman berikutnya, 
% seperti pada Tabel \ref{tbl:longtable1}.

% \begin{longtable}{@{\extracolsep{\fill}} l c r r}
% \caption{Comprehensive Data Table Example}\label{tbl:longtable1} \\
% \midrule
% \textbf{ID} & \textbf{Name} & \textbf{Score} & \textbf{Rank} \\
% \midrule
% \endfirsthead

% \caption[]{Comprehensive Data Table Example (lanjutan)} \\
% \midrule
% \textbf{ID} & \textbf{Name} & \textbf{Score} & \textbf{Rank} \\
% \midrule
% \endhead

% \midrule
% \multicolumn{4}{r}{\textit{Bersambung ke halaman berikutnya}} \\
% %\bottomrule
% \endfoot

% \midrule
% \endlastfoot

% % === Table Data ===
% 1 & Alice Smith & 89 & 5 \\
% 2 & Bob Johnson & 93 & 3 \\
% 3 & Carol Davis & 95 & 2 \\
% 4 & Daniel Wilson & 88 & 6 \\
% 5 & Eve Thompson & 97 & 1 \\
% 6 & Frank Brown & 85 & 7 \\
% 7 & Grace Lee & 91 & 4 \\
% 8 & Henry Miller & 80 & 9 \\
% 9 & Irene Garcia & 83 & 8 \\
% 10 & Jack Robinson & 78 & 10 \\
% % Repeat with more rows to make it long
% 11 & Kevin Harris & 76 & 11 \\
% 12 & Laura Martin & 75 & 12 \\
% 13 & Michael Clark & 74 & 13 \\
% 14 & Natalie Lewis & 73 & 14 \\
% 15 & Olivia Walker & 72 & 15 \\
% 16 & Peter Hall & 71 & 16 \\
% 17 & Quinn Allen & 70 & 17 \\
% 18 & Rachel Young & 69 & 18 \\
% 19 & Samuel King & 68 & 19 \\
% 20 & Tina Wright & 67 & 20 \\
% 21 & Uma Scott & 66 & 21 \\
% 22 & Victor Green & 65 & 22 \\
% 23 & Wendy Adams & 64 & 23 \\
% 24 & Xavier Nelson & 63 & 24 \\
% 25 & Yolanda Carter & 62 & 25 \\
% 26 & Zachary Perez & 61 & 26 \\
% 27 & Amelia Baker & 60 & 27 \\
% 28 & Benjamin Rivera & 59 & 28 \\
% 29 & Charlotte Rogers & 58 & 29 \\
% 30 & David Murphy & 57 & 30 \\
% 31 & Ethan Cooper & 56 & 31 \\
% 32 & Fiona Reed & 55 & 32 \\
% 33 & George Bailey & 54 & 33 \\
% 34 & Hannah Cox & 53 & 34 \\
% 35 & Isaac Howard & 52 & 35 \\
% 36 & Julia Ward & 51 & 36 \\
% 37 & Kyle Flores & 50 & 37 \\
% 38 & Lily Bell & 49 & 38 \\
% 39 & Mason Sanders & 48 & 39 \\
% 40 & Nora Patterson & 47 & 40 \\
% 41 & Owen Ramirez & 46 & 41 \\
% 42 & Penelope Torres & 45 & 42 \\
% 43 & Quentin Foster & 44 & 43 \\
% 44 & Rebecca Gonzales & 43 & 44 \\
% 45 & Sebastian Bryant & 42 & 45 \\
% 46 & Taylor Alexander & 41 & 46 \\
% 47 & Ursula Russell & 40 & 47 \\
% 48 & Vincent Griffin & 39 & 48 \\
% 49 & William Diaz & 38 & 49 \\
% 50 & Zoe Simmons & 37 & 50 \\
% % (You can easily extend this list to hundreds of rows)
% \end{longtable}

% \subsubsection{Beberapa Contoh Penulisan Rumus atau Persamaan Matematika Menggunakan LaTeX Termasuk Penomorannya}
% Contoh rumus matematika dapat ditulis seperti pada Persamaan \ref{eq:contoh1} di bawah ini. 
% Penomoran persamaan diletakkan di sebelah kanan, dan rumus ditulis dalam mode \textit{display math}.
% \begin{equation}
% E = mc^2
% \label{eq:contoh1}
% \end{equation}

% Contoh lain penulisan rumus matematika yang lebih kompleks dapat ditulis seperti pada Persamaan \ref{eq:rumus2}.

% \begin{align}
% f(x) &= ax^2 + bx + c \\
% f'(x) &= \frac{d}{dx}(ax^2 + bx + c) \notag \\ % tidak menampilkan nomor pada baris ini
%       &= 2ax + b \label{eq:rumus2}
% \end{align}

% Jika rumus terlalu panjang untuk ditulis dalam satu baris, gunakan lingkungan \textit{multline} seperti pada Persamaan \ref{eq:rumus3} di bawah ini.
% \begin{multline} 
% y = a_0 + a_1x + a_2x^2 + a_3x^3 + a_4x^4 + a_5x^5 + a_6x^6 + a_7x^7 \\
%     + a_8x^8 + a_9x^9 + a_{10}x^{10} \label{eq:rumus3}
% \end{multline}

% Jika ada penurunan rumus yang terdiri dari beberapa baris, namun tidak memerlukan penomoran pada setiap baris, gunakan lingkungan \textit{align*}, misalnya:

% \begin{align*} 
% S &= \sum_{i=1}^{n} i^2 \\
%   &= 1^2 + 2^2 + 3^2 + \cdots + n^2 \\
%   &= \frac{n(n + 1)(2n + 1)}{6}
% \intertext{Contoh lainnya adalah rumus untuk mencari nilai rata-rata fungsi $f(x)$ pada interval $[p, q]$:}
% \bar{f} &= \frac{1}{q - p} \int_{p}^{q} f(x) \, dx \\
%         &= \frac{1}{q - p} \int_{p}^{q} (ax^2 + bx + c) \, dx \\
%         &= \frac{1}{q - p} \left[ \frac{a}{3}x^3 + \frac{b}{2}x^2 + cx \right]_p^q \\
%         &= \frac{a(q^3 - p^3)}{3(q - p)} + \frac{b(q^2 - p^2)}{2(q - p)} + c \label{eq:rumus4}
% \end{align*}



% \subsection{Algoritma, Pseudocode, atau Kode}
% Contoh penulisan algoritma atau pseudocode dapat ditulis seperti pada Kode \ref{alg:contoh1} di bawah ini. 
% Gunakan paket \textit{listings} untuk menulis source code dalam bahasa pemrograman tertentu, seperti pada Kode \ref{lst:contoh1}. 


% % -- Example of pseudocode and source code listing --
% % -- Gunakan minipage agar listing tidak terpotong ke halaman berikutnya --
% \begin{minipage}{\textwidth} 
% \begin{lstlisting}[frame=lines, captionpos=t, caption={Contoh pseudocode}, label={alg:contoh1}]
% ALGORITHM HelloWorld
%    PRINT "Hello, World!"
% END ALGORITHM
% \end{lstlisting}
% \end{minipage}

% \begin{minipage}{\textwidth}
% \begin{lstlisting}[language=Python, frame=single, caption={Contoh source code Python}, captionpos=t, label={lst:contoh1}]
% def hello_world():
%     print("Hello, World!")       
% hello_world()
% \end{lstlisting}
% \end{minipage}


% \section{Beberapa Kesalahan Penulisan yang Sering Terjadi}
% \subsection{Penggunaan Kata "di mana" atau "dimana"}
% Banyak yang menuliskan kata "di mana" atau "dimana" sebagai pengganti kata "which" dalam bahasa Inggris. 
% Padahal, penggunaan kata "di mana" atau "dimana" tidak tepat dalam konteks tersebut. Demikian juga untuk kata serupa, misalnya "yang mana".
% Kata "di mana" atau "dimana" ini harus diganti dengan kata lain, seperti "dengan", "tempat", "yang", dan sebagainya tergantung kalimatnya.
% Penjelasan lengkap dapat dilihat pada \autocite{BPBI}.

% \subsection{Penggunaan Kata "sedangkan" dan "sehingga"}

% \begin{table}[t]
%   \begin{tabular}{|c|l|l|}
%   \hline
%   Kata & Salah & Benar \\ \hline
%   sedangkan & \begin{tabular}[c]{@{}c@{}}Sedangkan sistem lama masih\\ digunakan oleh banyak pengguna.\end{tabular} & \begin{tabular}[c]{@{}c@{}}Sistem lama masih digunakan\\ oleh banyak pengguna,\\ sedangkan sistem baru belum siap.\end{tabular} \\ \hline
%   sehingga & \begin{tabular}[c]{@{}c@{}}Sehingga sistem lama masih\\ digunakan oleh banyak pengguna.\end{tabular} & \begin{tabular}[c]{@{}c@{}}Sistem lama masih digunakan\\ oleh banyak pengguna sehingga\\ sistem baru belum siap.\end{tabular} \\ \hline
%   \end{tabular}
%   \caption{Contoh penggunaan kata "sedangkan" dan "sehingga"}
%   \label{tbl:sedangkan_sehingga}
% \end{table}

% Kata "sedangkan" dan "sehingga" adalah kata hubung atau konjungsi. 
% Konjungsi adalah kata atau ungkapan yang menghubungkan satuan bahasa 
% (kata, frasa, klausa, dan kalimat). 
% Konjungsi dapat dibagi menjadi konjungsi intrakalimat dan antarkalimat.  
% Kata "sedangkan" menghubungkan dua klausa yang bersifat kontrasif, 
% sedangkan "sehingga" menghubungkan dua klausa yang bersifat kausal. 
% Dalam ragam formal, kata hubung “sedangkan” dan “sehingga” hanya dapat digunakan 
% sebagai konjungsi intrakalimat sehingga kedua konjungsi itu \textbf{tidak dapat diletakkan pada awal kalimat}.
% Selain itu, penggunaan kata "sedangkan" harus didahului oleh koma (,), sedangkan kata "sehingga" tidak perlu didahului oleh koma (,).
% Contoh penggunaan yang benar dan salah dapat dilihat pada Tabel \ref{tbl:sedangkan_sehingga}.


% \subsection{Penggunaan Istilah yang Tidak Baku}
% Ada beberapa istilah yang sering digunakan dalam pembicaraan sehari-hari, tetapi tidak baku dalam penulisan ilmiah.
% Beberapa istilah tersebut antara lain:
% \begin{enumerate}
%   \item analisa $\rightarrow$ analisis
%   \item eksisting atau existing $\rightarrow$ yang ada atau saat ini
%   \item bisnis proses $\rightarrow$ proses bisnis
%   \item user $\rightarrow$ pengguna
%   \item system $\rightarrow$ sistem
%   \item database $\rightarrow$ basis data
%   \item aktifitas $\rightarrow$ aktivitas
%   \item efektifitas $\rightarrow$ efektivitas
%   \item sosial media $\rightarrow$ media sosial
% \end{enumerate}

% \subsection{Pemisah Desimal dan Ribuan}
% Tanda pemisah desimal dalam bahasa Indonesia adalah tanda koma, contoh:
% \begin{enumerate}
%   \item (Salah) Akurasi naik menjadi 50.6\% 
%   \item (Benar) Akurasi naik menjadi 50,6\% 
% \end{enumerate}

% \subsection{Daftar atau \textit{List}}
% Ada beberapa aturan penulisan daftar atau \textit{list} yang perlu diperhatikan, antara lain:
% \begin{enumerate}[a)]
% \item Jika memungkinkan, hindari penggunaan “bullet points” atau sejenisnya. Sebaiknya, gunakan angka (1, 2, 3, ...) atau huruf (a, b, c, …). Dengan demikian, pembaca dapat dengan mudah melihat jumlah \textit{item} atau \textit{list}. 
% \item Jika dalam daftar hanya ada satu item, tidak perlu menggunakan nomor urut.
% \item Penjelasan atau deskripsi suatu item sebaiknya menyatu dengan judul item tersebut, tidak berbeda halaman. Contoh yang salah: judul item ada di halaman 10, namun deskripsinya di halaman 11. Sebaiknya pindahkan judul tersebut ke halaman 11.
% \item Jika penjelasan atau deskripsi suatu item cukup panjang, misalnya lebih dari 1 halaman atau terdiri atas beberapa paragraf, sebaiknya setiap item tersebut dijadikan judul subbab, kecuali jika level subbab sudah mencapai level 4. 
% \end{enumerate}

% \subsection{Penggunaan Kata "masing-masing" dan "setiap"}
% Kata “masing-masing” digunakan di belakang kata yang diterangkan, misalnya 
% "Setiap proses menggunakan algoritma masing-masing". Kata “tiap-tiap” atau “setiap”
% ditempatkan di depan kata yang diterangkan, misalnya
% "Setiap proses menggunakan algoritma tertentu".